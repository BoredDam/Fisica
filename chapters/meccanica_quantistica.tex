\chapter{Introduzione alla Meccanica Quantistica}

La meccanica quantistica è una teoria fondamentale della fisica che descrive il comportamento dei sistemi fisici a scala microscopica, come elettroni, fotoni, atomi e molecole. Essa nasce all’inizio del XX secolo come risposta a una profonda crisi della fisica classica, che, pur avendo ottenuto enormi successi nella descrizione dei fenomeni macroscopici, si dimostrò incapace di spiegare correttamente numerosi risultati sperimentali osservati a piccole scale.

La meccanica classica si fonda su una visione deterministica della natura: una volta note le condizioni iniziali di un sistema, le leggi del moto permettono di determinarne univocamente l’evoluzione temporale. Tuttavia, quando si analizzano fenomeni che coinvolgono la radiazione elettromagnetica e la struttura della materia su scale atomiche, questa descrizione risulta inadeguata. In particolare, emergono discrepanze tra le previsioni teoriche e i dati sperimentali che non possono essere risolte mediante semplici correzioni dei modelli classici.

La meccanica quantistica introduce quindi un cambiamento radicale di paradigma. In questo nuovo quadro teorico, lo stato fisico di un sistema non è più descritto in termini di traiettorie ben definite, ma attraverso una funzione d’onda, dalla quale è possibile ricavare solo le probabilità dei possibili risultati di una misura. L’indeterminazione e la probabilità non rappresentano limiti sperimentali o imperfezioni nella misura, ma sono caratteristiche intrinseche della descrizione quantistica della natura.

Nel seguito, ricostruiremo il percorso storico e concettuale che ha condotto alla nascita della meccanica quantistica, partendo dall’analisi delle onde classiche e dalla crisi della fisica classica, fino all’introduzione della funzione d’onda e dell’equazione di Schrödinger, che costituiscono il cuore del formalismo quantistico.

\section{La crisi della Fisica Classica}

Alla fine del XIX secolo, la fisica classica appariva come una teoria sostanzialmente completa. La meccanica newtoniana, l’elettromagnetismo di Maxwell e la termodinamica fornivano una descrizione estremamente accurata di una vasta gamma di fenomeni naturali. Tuttavia, proprio nello studio dei sistemi che coinvolgono la radiazione elettromagnetica e la materia a scale microscopiche iniziarono ad emergere risultati sperimentali incompatibili con le previsioni teoriche.

\subsection{Il problema della radiazione del corpo nero}
Un esempio emblematico di questa crisi è rappresentato dallo studio della radiazione emessa da un corpo nero. 
\begin{quote}
Un corpo nero è un sistema ideale che assorbe completamente la radiazione elettromagnetica incidente, indipendentemente dalla lunghezza d’onda, e che emette radiazione con uno spettro che dipende unicamente dalla sua temperatura. 
\end{quote}
Le misure sperimentali dello spettro di emissione mostravano una distribuzione ben definita dell’energia in funzione della frequenza, con un massimo a una frequenza caratteristica che cresce all’aumentare della temperatura.

\begin{figure}[htbp]
    \centering
    \includegraphics[width=0.8\textwidth]{images/corpo_nero.png}
    \caption{Modello di corpo nero. Una cavità con pareti interne annerite e un piccolo forellino assorbe quasi completamente la radiazione elettromagnetica incidente. La radiazione che emerge dal forellino è indipendente dal materiale delle pareti e dipende unicamente dalla temperatura $T$ del sistema, realizzando una buona approssimazione di un corpo nero ideale. A sinistra è mostrato lo spettro di emissione per diverse temperature $T_1 < T_2 < T_3$, evidenziando lo spostamento del massimo verso lunghezze d’onda minori all’aumentare della temperatura.}
    \label{fig:corpo_nero}
\end{figure}

La fisica classica tentò di spiegare questo fenomeno modellando la radiazione elettromagnetica all’interno del corpo nero come un insieme di onde stazionarie. Secondo la teoria classica, l’energia associata a ciascuna modalità di oscillazione poteva assumere qualsiasi valore continuo, portando alla cosiddetta \textit{catastrofe ultravioletta}: la previsione teorica indicava che l’energia totale emessa dal corpo nero sarebbe dovuta essere infinita, in netto contrasto con i risultati sperimentali che mostravano un’energia finita e ben definita.

\paragraph{Un esempio di corpo nero: il Sole.}
Il sole è un esempio naturale di corpo nero approssimato. La sua superficie assorbe quasi completamente la radiazione elettromagnetica di qualsiasi lunghezza d'onda incidente su di esso a qualsiasi angolo, e la radiazione emessa dipende principalmente dalla sua temperatura superficiale, che è di circa 5778 K. Lo spettro di emissione del sole segue approssimativamente la legge di Planck per un corpo nero, con un picco di emissione nella regione visibile dello spettro elettromagnetico, il che spiega perché il sole appare luminoso per l'occhio umano.

\subsection{Spettro di corpo nero secondo la fisica classica}

Introduciamo la quantità spettrale
\begin{equation}
\boxed{
\varepsilon(\nu) \equiv \frac{dE}{dt\, dA\, d\nu},
}
\end{equation}
\addcontentsline{equ}{myequations}{Spettro di corpo nero secondo la fisica classica}
dove $dE$ è l'energia emessa dal corpo nero tramite radiazione elettromagnetica
nell'intervallo di frequenze $[\nu,\nu+d\nu]$, per unità di tempo $dt$ e per unità
di area emissiva $dA$.
In altre parole, $\varepsilon(\nu)$ rappresenta l'energia emessa \emph{per unità di tempo,
unità di area e unità di frequenza} (notazione delle slide).

L'idea classica consiste nel modellare la radiazione elettromagnetica all'interno di una cavità
(che realizza un corpo nero) come un insieme di \emph{onde stazionarie}.
Le onde stazionarie ammissibili sono quantizzate solo \emph{geometricamente} (condizioni al contorno),
e per ciascun intervallo di frequenze $[\nu,\nu+d\nu]$ esiste un numero di modi proporzionale a $\nu^2$.
Questo porta a scrivere la densità spettrale emessa come
\[
\varepsilon(\nu)\, d\nu = \frac{8\pi \nu^2}{c^3}\,\langle E\rangle\, d\nu,
\]
dove:
\begin{itemize}
    \item $\frac{8\pi \nu^2}{c^3}\,d\nu$ è il contributo dovuto al \emph{numero di modi}
    (onde stazionarie) del campo elettromagnetico nell'intervallo $[\nu,\nu+d\nu]$;
    \item $\langle E\rangle$ è l'energia media associata a ciascun modo a temperatura $T$.
\end{itemize}

Per determinare $\langle E\rangle$, la fisica classica applica il principio di equipartizione
dell'energia.
Ogni modo del campo elettromagnetico possiede due gradi di libertà (due polarizzazioni),
e quindi l'energia media per modo risulta
\[
\langle E\rangle
= 2\times \frac{1}{2}\, k_B T
= k_B T.
\]
Sostituendo nella relazione precedente si ottiene la legge di Rayleigh--Jeans:
\begin{equation}
\boxed{
\varepsilon(\nu)\, d\nu
= \frac{8\pi \nu^2}{c^3}\, k_B T\, d\nu,
}
\qquad \text{(Rayleigh--Jeans, 1900--1905).}
\end{equation}
\addcontentsline{equ}{myequations}{Legge di Rayleigh--Jeans per lo spettro di corpo nero}


Questa previsione classica descrive correttamente il comportamento a basse frequenze,
ma cresce come $\nu^2$ per $\nu \to \infty$ e quindi porta a un'energia emessa divergente
alle alte frequenze (catastrofe ultravioletta), in disaccordo con i dati sperimentali
misurati (ad es.\ Lummer e Pringsheim, 1899).

\begin{figure}[htbp]
    \centering
    \includegraphics[width=0.5\textwidth]{images/confronto_spettri_corpo_nero.png}
    \caption{Confronto tra lo spettro di emissione del corpo nero misurato sperimentalmente (Lummer e Pringsheim, 1899) e la previsione della fisica classica. Le curve nere rappresentano l’andamento sperimentale dell’emissione per diverse temperature, mentre la curva arancione indica la legge di Rayleigh--Jeans. Quest’ultima descrive correttamente il comportamento alle grandi lunghezze d’onda, ma diverge alle piccole lunghezze d’onda, evidenziando il fallimento della descrizione classica noto come catastrofe ultravioletta.}
    \label{fig:confronto_spettri_corpo_nero}
\end{figure}


\subsection{Spettro di Planck e superamento della descrizione classica}

Nella descrizione classica dello spettro di corpo nero, la densità spettrale di energia è scritta nella forma
\[
\varepsilon(\nu)\, d\nu = \frac{8\pi \nu^2}{c^3}\,\langle E\rangle\, d\nu,
\]
dove il fattore $\frac{8\pi \nu^2}{c^3}$ rappresenta il numero di modi (onde stazionarie) del campo elettromagnetico nell’intervallo di frequenze $[\nu,\nu+d\nu]$, mentre $\langle E\rangle$ è l’energia media associata a ciascun modo.

Il fallimento della fisica classica non risiede dunque nel conteggio dei modi del campo elettromagnetico, che rimane invariato, bensì nell’espressione dell’energia media $\langle E\rangle$. In ambito classico, applicando il principio di equipartizione dell’energia, si ottiene $\langle E\rangle = k_B T$, il che conduce alla legge di Rayleigh--Jeans e alla divergenza dello spettro alle alte frequenze (catastrofe ultravioletta).

La svolta introdotta da Planck consiste in una modifica radicale dell’ipotesi sull’energia dei modi del campo elettromagnetico. In particolare, Planck postulò che l’energia di ciascun modo di frequenza $\nu$ non potesse variare in modo continuo, ma fosse quantizzata secondo la relazione
\[
E_n = n\, h \nu, \qquad n = 0,1,2,\dots
\]
dove $h = 6.626 \times 10^{-34} \, \text{J s}$, oppure in forma ridotta,$ \hbar = \frac{h}{2\pi}$, è la costante di Planck.

Come conseguenza di questa ipotesi, il valore medio dell’energia di un singolo modo del campo elettromagnetico in equilibrio termico alla temperatura $T$ risulta
\[
\langle E\rangle = \frac{h\nu}{e^{h\nu / k_B T} - 1},
\]
in disaccordo con il principio di equipartizione dell’energia della fisica classica.

Sostituendo questa espressione di $\langle E\rangle$ nella formula generale per la densità spettrale, si ottiene lo \emph{spettro di Planck}:
\begin{equation}
\boxed{
\varepsilon(\nu)\, d\nu
= \frac{8\pi \nu^2}{c^3}\,
\frac{h\nu}{e^{h\nu / k_B T} - 1}\, d\nu.
}
\end{equation}
\addcontentsline{equ}{myequations}{Spettro di Planck per il corpo nero}

Lo spettro di Planck descrive correttamente l’andamento sperimentale dell’emissione di un corpo nero a tutte le frequenze: esso coincide con la legge classica di Rayleigh--Jeans nel limite delle basse frequenze, mentre alle alte frequenze presenta un decadimento esponenziale che elimina la divergenza ultravioletta.

Nel limite formale $h \to 0$, oppure per $h\nu \ll k_B T$, l’energia media tende al valore classico,
\[
\langle E\rangle \to k_B T,
\]
e la descrizione di Planck si riduce a quella prevista dalla fisica classica. Questo mostra come la meccanica quantistica generalizzi la teoria classica, recuperandone i risultati nel limite opportuno.

